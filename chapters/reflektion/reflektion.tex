\chapter{Reflektion und Ausblick}
Die vorliegende Bachelorarbeit befasst sich mit der Frage
\enquote{Wie kann ein System entwickelt werden, mit dem verschiedene DL-Modelle
verwaltet werden können, welches modular und einfach erweiterbar ist
und Modellergebnisse über gemeinsame Schnittstellen im Netz kommuniziert?}
Um diese Frage zu beantworten, wurden
die Grundlagen von Deep Learning untersucht
und den Prozess beschrieben, wie mit TensorFlow und Keras
ein neuronales Netz erstellt und trainiert werden kann.
Anhand dieser Überlegung konnte schließlich die Architektur einer Anwendung
vorgestellt werden, mit der ein neuronales Netz in der Praxis eingesetzt wird.\\[8pt]
Das Modell wurde in ein mobil- und TPU-kompatibles Format umgewandelt
und anschließend miteinander vergleichen.
Mit \autoref{fig:model-comparison} ist zu erkennen,
dass hierdurch keine oder nur kaum zu bemerkende Genauigkeitsverluste
entstanden sind.
Jedoch wird nach \autoref{fig:sine_model_inference_time}
das Ergebnis festgehalten,
dass zumindest bei kleinen Modellen und bestimmten Netzarchitekturen
die Inferenzgeschwindigkeit
auf der Edge TPU nicht unbedingt schneller ist als die gleiche Ausführung auf der CPU.
Es wird deshalb der Entschluss gefasst,
ein entwickeltes Netz immer zusammen mit der verwendeten Hardware zu testen,
um das mit den besten Ergebnissen auszuwählen.\\[8pt]
Die vorgestellte Anwendung \eqref{fig:app-architecture} besteht aus mehreren kleinen
Einheiten. Im Rahmen der Zielsetzung des System als ein Demoprojekt zu dienen
wurde eine Möglichkeit demonstriert,
wie diese mit Docker auf einem Gerät wie dem Siemens
SIMATIC IOT2050 bereitgestellt wird.\\[8pt]
Weitere Forschung und Entwicklung können auf diesen Ergebnissen
aufbauen. Es ist gut möglich die Anwendung zu erweitern oder
auch einzelne Dienste auszuwechseln.
Aktuell ist die Coral-Anwendung in Python geschrieben.
Um den Mehraufwand zu reduzieren, der dadurch entsteht ein
Python-Programm auszuführen, könnte diese in ein \cpp{}-Projekt
umgewandelt werden.
Dies sollte nicht nur die Leistung des Dienstes verbessern, sondern
auch die Imagegröße reduzieren.