\chapter{Reflektion und Ausblick}
Die vorliegende Bachelorarbeit befasst sich mit der Frage
\enquote{Wie kann ein System entwickelt werden, mit dem verschiedene DL-Modelle
verwaltet werden können, welches modular und einfach erweiterbar ist
und Modellergebnisse über gemeinsame Schnittstellen im Netz kommuniziert?}
Um diese Frage zu beantworten, wurden
die Grundlagen von Deep Learning untersucht
und den Prozess beschrieben, wie mit TensorFlow und Keras
ein neuronales Netz erstellt und trainiert werden kann.
Das Modell wurde in ein mobil- und TPU-kompatibles Format umgewandelt
und anschließend miteinander vergleichen.
Hierbei konnte das Ergebnis festgehalten werden,
dass zumindest bei kleinen Modellen und bestimmten Netzarchitekturen
die Inferenzgeschwindigkeit
auf der Edge TPU nicht unbedingt schneller ist als die gleiche Ausführung auf der CPU.