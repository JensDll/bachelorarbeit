\chapter{Einleitung}
Hören die meisten Personen die Begriffe \enquote{Künstliche Intelligenz}
und \enquote{Maschinelles Lernen} stellen sie sich Roboter vor:
treue Diener, die all deine Aufgaben übernehmen oder eine tödliche Arme
von superintelligenten Cyborgs im Kampf gegen die Menschheit.
Künstliche Intelligenz (kurz KI) ist jedoch schon lange nicht mehr nur eine futuristische
Fantasie und Teil von Science-Fiction-Filmen; sie wird bereits großflächig
verwendet und findet Einsatz in einer Vielzahl von Anwendungen aus Bereichen
in nahe zu allen Teilen der Wirtschaft. Andrew Ng,
welcher unteranderem durch seine beliebten Onlinekurse zum Thema
Maschinellen Lernen (ML) und Deep Learning (DL)
bekannt ist,
\footnote{\url{https://www.coursera.org/instructor/andrewng}}
vergleicht KI mit der neuen Elektrizität.
\begin{aquote}{Andrew Ng \parencite{online:ai-andrew-ng}}
  \enquote{\textit{Just as electricity transformed almost everything 100 years ago,
      today I actually have a hard time thinking
      of an industry that I don’t think AI will
      transform in the next several years}}
\end{aquote}
Die Anwendung, mit der ML erstmalig öffentliche Aufmerksamkeit
erlangt und das täg\-liche Leben vieler Menschen beeinflusst, stammt aus den 90er-Jahren:
Die Rede ist von den E-Mail-Spamfiltern \parencite[1]{book:hands-on-ml}.
Es handelt sich um eine scheinbar leichte Aufgabe, welche
mit traditioneller Programmierung aber dennoch nur schwer gelöst werden kann.
Ein klassisches Programm besteht aus vielen statischen Regeln,
im Fall des Spamfilters können diese dazu dienen, auffällige Schlüsselwörter und
Satzteile zu erkennen (z.\,B. \enquote{Kreditkarte}, \enquote{konstenlos}, \enquote{kaufen} usw).
Ein statisches Programm ist nicht gut dafür geeignet, ein dynamisches Problem
zu beschreiben. Die Absender der Spamnachrichten könnten erkennen, welche E-Mails blockiert werden
und diese leicht abändern. Arbeiten sie so um das Problem herum,
müssen immer wieder neue Regeln im Programm aufgenommen werden. Ein Spamfilter hingegen,
welcher auf Maschinellen Lernen basiert, kann diese Regeln
selbstständig erkennen.
Im Training verwendet der Algorithmus hierfür Beispiel E-Mails
(z.\,B. diese, die vom Benutzer markiert wurden), diese Daten nennt man dann den
Trainingsdatensatz.
Es folgt ein Programm, welches weitaus kürzer,
einfacher zu verwalten und wahrscheinlich auch präziser ist.

\section{Was ist Maschinelles Lernen?}
Maschinelles Lernen ist die Wissenschaft (und Kunst) der Programmierung,
die es dem Computer ermöglicht, von Daten zu lernen. Arthur Samuel und Tom Mitchell
haben im Jahr 1959 und 1997 eine allgemeine und formale Definition gegeben \parencite[2]{book:hands-on-ml}:
\begin{aquote}{Arthur Samuel, 1959}
  \enquote{\textit{Machine Learning is the field of study that gives computers
      the abilty to learn without being explicitly programmed.}}
\end{aquote}
\begin{aquote}{Tom Mitchell, 1997}
  \enquote{\textit{A computer program is said to learn from experience
      $E$ with respect to some task \,$T$ and some performance measure $P$,
      if its performance on $T$, as measured by $P$, improves with experience $E$.}}
\end{aquote}
